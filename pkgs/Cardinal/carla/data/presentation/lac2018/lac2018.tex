\documentclass[11pt,a4paper]{article}
\usepackage{lac2018}
\sloppy
\newenvironment{contentsmall}{\small}

\title{Carla Plugin Host - Feature overview and workflows, c-base Berlin}

%see lac2018.sty for how to format multiple authors!
\author
{Filipe Coelho
\\ falkTX, Linux Audio Developer
\\ Berlin, Germany,
\\ falktx@falktx.com
}



\begin{document}
\maketitle


\begin{abstract}
\begin{contentsmall}
Carla is a fully-featured audio plugin host with support for many plugin formats,
featuring automation of plugin parameters via MIDI CC, remote control over OSC, among others.
This workshop plans to give a quick overview of Carla and go through some workflows together with the audience.
\end{contentsmall}
\end{abstract}

\keywords{
\begin{contentsmall}
Plugin, Host, Modular, MIDI, OSC, Rack, Patchbay
\end{contentsmall}
}

\section{Introduction}

The first part of the workshop is a quick personal introduction, followed by a small musical demo.

The demo features a MIDI sequencer to generate events, but the entire output sound comes from Carla.
It serves as a demonstration of what we can with it.

\section{Overview}

After showing a demo song, the workshop continues by a quick overview of Carla's features and its graphical interface.

This will allow for those that don't know Carla yet to easily catch up on its current status.

\subsection{Features}

First, we describe the main Carla features, to inform those that don't know the application yet.

\subsection{Interface}

Next, we go through the most important elements of Carla's UI and explain what they do, and what they mean.

We also describe the possible settings, and do a local scan for plugins (so we can actually use them).

\section{Workflows}

This is the main content of the workshop.

We will go through a few Carla setups to cover as many use-cases as possible.

\subsection{The First Sound}

To get the audience acquainted with Carla, we will first load a few plugins and have them make some sound.
We introduce managing plugins and connections here.
If possible, we demonstrate the use of a MIDI keyboard together with Carla.

\subsection{Rack Mode}

One of the main features of Carla is the Rack mode.

We will describe how it works, and how we can use it in creative ways to produce sounds using multiple plugins.

\subsection{Sequencing MIDI}

In a modular host, we can create sound without user input by using MIDI generator plugins.

We will demonstrate two of such plugins. and use them to drive a basic drum kit and synth.

\subsection{Modular Side-chain}

Modular setups make it easy to setup side-chains.

We will quickly go through one of these, using what we learned so far.

\subsection{Carla as a plugin}

Carla works as a plugin, not just a standalone application.

We will show some possible use-cases for this feature, showing Carla running inside Carla and in regular DAWs.

\subsection{Remote OSC Control}

OSC is a common protocol used to control audio applications remotely.
Carla-Control can be used to control a remote Carla instance over the network.

We will connect two Carla instances together, running on different systems.

\section{Questions}

We reserve a space at the end of the workshop for questions from the audience.

\section{Conclusions}

We conclude the workshop with a thank you to the audience.

\end{document}
